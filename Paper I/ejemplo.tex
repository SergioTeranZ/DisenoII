\documentclass{ci5652}
\usepackage{graphicx,amssymb,amsmath}
\usepackage[utf8]{inputenc}
\usepackage[spanish]{babel}
\usepackage{hyperref}
\usepackage{subfigure}
\usepackage{paralist}
\usepackage[ruled,vlined,linesnumbered]{algorithm2e}

%----------------------- Macros and Definitions --------------------------

% Add all additional macros here, do NOT include any additional files.

% The environments theorem (Theorem), invar (Invariant), lemma (Lemma),
% cor (Corollary), obs (Observation), conj (Conjecture), and prop
% (Proposition) are already defined in the ci5652.cls file.

%----------------------- Title -------------------------------------------

\title{Solución al problema de la mochila usando búsqueda local}

\author{Marialicia Suárez
        \and
        Sergio Teŕan}

%------------------------------ Text -------------------------------------

\begin{document}
\thispagestyle{empty}
\maketitle


\section*{Resúmen}
El problema de la mochila surge cuando hay una asignación de recursos con restricciones de costo. Cada ítem tiene un costo y un valor, se busca obtener el mayor valor posible con un presupuesto dado. El termino \textit{Problema de la mochila} evoca la imagen de un viajero con una mochila de tamaño fijo que debe llenar solo con los ítem mas útiles. Este problema se clasifica como un problema de optimización combinatoria y, computacionalmente, se categoriza como NP-Completo~\cite{c_resumen_01}. Este articulo propone solucionar el problema planteado usando una función heurística de búsqueda local.

\section*{Palabra Clave}


\begin{abstract}
The Knapsack problem arises when there is an allocation of resources with cost constraints. Each item has a cost and a value, it seeks to obtain the highest value possible with a given budget. The term \textit{Knapsack Problem} evokes the image of a traveler with a backpack fixed size that should be filled only with the most useful item. This problem is classified as a combinatorial optimization problem and, computationally, categorized as NP-Hard~\cite{c_resumen_01}. This article proposes to solve the problem raised using a local search heuristic function.
\end{abstract}

\section{Una sección cualquiera}
Citan así.

\begin{algorithm}
 \DontPrintSemicolon
 \vspace*{0.1cm}
 \KwIn{Descripcion}
 \KwOut{Descripcion}
 Primer paso\;
 Segundo\;
 \ForEach{$i = 1\dots n$}{
  \If{Alguna condición}{
   Algo aqui\;
   }
 }
 \KwRet{Valor}
 \vspace*{0.1cm}
 \caption{Nombre}
\end{algorithm}


\section{Otra sección}

Hola.

\section*{Conclusiones}

Aquí concluyen.

%---------------------------- Bibliography -------------------------------

% Please add the contents of the .bbl file that you generate,  or add bibitem entries manually if you like.
% The entries should be in alphabetical order
\small
\bibliographystyle{abbrv}

\begin{thebibliography}{99}

\bibitem{c_resumen_01}
Papadimitriou, C. H., Steiglitz, K. 
\newblock Combinatorial Optimization: Algorithms and Complexity. 
\newblock {\em New York: Englewood Cliffs}, 1998.

\end{thebibliography}


\newpage
\section*{Apéndice}

Bla.

\end{document}
